\newpage
{\Huge \bf Abstract}
\vspace{24pt} 

This project investigates the use of model-based reinforcement learning (RL) in the domain of computer systems, specifically, that of optimising deep learning models by applying transformations to the network which is represented as a graph. Reducing the hardware resource requirements is a open, active research challenge and one direction is to design optimal heuristic rules. In this work, we investigated the use of RL agents that can learn to perform optimal substitutions, without the need of expert human heuristics to achieve a high level of performance. Recent work has aimed to apply reinforcement learning to computer systems with some success, especially using model-free RL techniques. However, model-based methods have seen an increased focus of research recently, model-based reinforcement learning can be used to learn a model of the environment that can be leveraged to train an agent inside the learned world-model---increasing sample efficiency. Furthermore, when using a world model as the environment, batch rollouts can occur safely in parallel and, especially in systems environments, it overcomes the possible latency impact of stepping a system environment that can take orders of magnitude longer to perform an action compared to a simple emulators for video games. This dissertation examines both the prior work for optimising deep learning models and the applicability of reinforcement learning to the problem.

\vspace*{\fill}
