\newpage
{\Huge \bf Abstract}
\vspace{24pt} 

% This project investigates the use of model-based reinforcement learning (RL) in the domain of computer systems, specifically, that of optimising deep learning models by applying transformations to the computation graph to minimise the runtime cost on hardware devices. Recent work has aimed to apply reinforcement learning to computer systems with some success, especially with using model-free RL techniques. However, more recently, model-based methods has seen an increased focus of research as model-based reinforcement learning can learn a model of the environment, such that an agent can take actions inside the learned world-model to train more efficiently; environment rollouts can occur safely in parallel and, especially in systems environments, it circumvents the possible latency impact of stepping a system environment that can take orders of magnitude longer to perform an action compared to a video game emulator for example. This dissertation examines both the prior work for optimising deep learning models and the applicability of reinforcement learning to the problem.

\vspace*{\fill}
